%using document class from KOMA-Script
\documentclass{scrartcl}
\title{My Machine Learning Guide}
\subtitle{}
\date{}
\author{Fawaz Shah}

% Packages for adding hyperlinks to table of contents
\usepackage{color}   %May be necessary if you want to color links
\usepackage{hyperref}
\hypersetup{
    colorlinks=true, %set true if you want colored links
    linktoc=all,     %set to all if you want both sections and subsections linked
    linkcolor=black,  %choose some color if you want links to stand out
}

%package that allows aligned equations
\usepackage{amsmath}

%package that allows notation for extra mathematical symbols
\usepackage{amssymb}

%new commands for popular sets for ease of use
\newcommand{\R}{\mathbb{R}}
\newcommand{\N}{\mathbb{N}}
\newcommand{\Z}{\mathbb{Z}}
\newcommand{\Q}{\mathbb{Q}}
\newcommand{\C}{\mathbb{C}}

% renaming command to writing vectors in bold notation
\renewcommand{\vec}[1]{\mathbf{#1}}

%package for managing images
\usepackage{graphicx}
\graphicspath{ {../img/} }

%package for managing hyperlinks
\usepackage{hyperref}

%package for added blue colored boxes
\usepackage[most]{tcolorbox}

%package that allowed removing indent from enumerate environment
\usepackage{enumitem}

%package that allows for negating nearly any symbol
\usepackage{centernot}

%enlarging line spacing in tables
\renewcommand{\arraystretch}{2}

\begin{document}
\large
\maketitle
\tableofcontents
\newpage

\part{Supervised Learning}

Supervised learning methods can be split into two types: regression problems and classification problems.
\\\\
In all supervised learning methods we have a \textbf{training set}, which is a collection of data which we already know the right answers for.

\section{Regression}

Regression methods are used for predicting continuous data.

\subsection{Cost functions}

A popular cost function is the \textbf{mean squared error}.
\begin{equation}
E = \frac{1}{2n} \sum^{n}_{i = 1} (f(x_{i}) - y_{i})^{2}
\end{equation}

\subsection{Linear regression}

We try to fit a straight line to some data.\\
We have two variables, $ x $ and $ y $.\\
Our line of best fit will be of the form:
\begin{equation}
f(p_{0}, p_{1}) = p_{0}x + p_{1}
\end{equation}
where we must choose the optimum parameters ($ p_{0} $, $ p_{1} $) such that our cost function is minimized.
\\\\
This is achieved by gradient descent.
\begin{equation}
p_{i} := p_{i} - \alpha \frac{\partial}{\partial p_{i}} f(p_{0}, p_{1})
\end{equation}
The $ p_{i} $ will converge to their optimal value. $ \alpha $ is what is called the \textbf{learning rate}, which decides how fast the parameters converge.
\\\\
Deciding the optimal learning rate for the problem is tricky. If the learning rate is too slow then it will take a long time for the parameters to converge. If the learning rate is too fast then the parameters may overshoot the minimum point.


\subsection{Logistic regression}

1 variable\\
More than 1 variable

\section{Classification}

Classification methods are used for predicting discrete data. We try to 
group data into categories.

\subsection{PCA - Principal Component Analysis}

Dimensionality reduction

\newpage
\part{Unsupervised Learning}

\section{Clustering}

\subsection{K-means clustering}

\newpage
\part{Reinforcement Learning}

\end{document}